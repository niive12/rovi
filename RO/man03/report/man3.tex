\documentclass[a4paper,10pt]{article}
\usepackage[utf8]{inputenc}
\usepackage{subcaption}

\title{ROVI - Mandatory project 3}
\author{Nikolaj Iversen and Lukas Schwartz}

\begin{document}

\maketitle

\section{Find Wrot}

Given the frames:
\[ 
F^0 = 
\left[
\begin{array}{*{3}{p{0.5cm}}|c}
    1 & 0 & 0 & 15 \\
    0 & 1 & 0 &  8 \\
    0 & 0 & 1 &  3 \\\hline
    0 & 0 & 0 &  1 \\
\end{array}
\right]
\]
\[ 
F^1 = 
\left[
\begin{array}{*{3}{p{0.5cm}}|c}
    0 & 1 & 0 & 10 \\
   -1 & 0 & 0 &  4 \\
    0 & 0 & 1 &  2 \\\hline
    0 & 0 & 0 &  1 \\
\end{array}
\right]
\]
\[ 
F^2 = 
\left[
\begin{array}{*{3}{p{0.5cm}}|c}
    0 & 0 & 1 &  6 \\
   -1 & 0 & 0 &  0 \\
    0 &-1 & 0 & -2 \\\hline
    0 & 0 & 0 &  1 \\
\end{array}
\right]
\]

The rotation \(W_{rot}\) is found using equation \ref{eq:wrot_general}. 

\begin{equation}
 R_d = 
 \left[ 
 \begin{array}{c}
  R_{32} - R_{23}\\
  R_{13} - R_{31}\\
  R_{21} - R_{12}\\
 \end{array}
 \right]
\end{equation}

\begin{equation}
 W_{rot}(R) = \frac{cos^{-1} \left( \frac{R_{11}+R_{22}+R_{33}-1}{2} \right) }{||R_d||} \cdot R_d \label{eq:wrot_general}
\end{equation}

\(W_{rot}(\left[R^0\right]^T R^1)\) and \(W_{rot}(\left[R^1\right]^T R^2)\) should be found.

\[
\left[R^0\right]^T R^1
=
\left[\begin{array}{*{3}{p{0.5cm}}}
  1 & 0 & 0\\
  0 & 1 & 0\\
  0 & 0 & 1\\
\end{array}\right]
\left[\begin{array}{*{3}{p{0.5cm}}}
  0 & 1 & 0\\
 -1 & 0 & 0\\
  0 & 0 & 1\\
\end{array}\right]
=
\left[\begin{array}{*{3}{p{0.5cm}}}
  0 & 1 & 0\\
 -1 & 0 & 0\\
  0 & 0 & 1\\
\end{array}\right]
\]

\[
 W_{rot} = 
 \frac{cos^{-1}(0)}{2} *  
 \left[\begin{array}{c}
  0\\
  0\\
  -2\\
 \end{array}\right] =
 \left[\begin{array}{c}
  0\\
  0\\
  -\pi/2\\
 \end{array}\right] 
\]

\[
\left[R^1\right]^T R^2
=
\left[\begin{array}{*{3}{p{0.5cm}}}
  0 &-1 & 0\\
  1 & 0 & 0\\
  0 & 0 & 1\\
\end{array}\right]
\left[\begin{array}{*{3}{p{0.5cm}}}
  0 & 0 & 1\\
 -1 & 0 & 0\\
  0 &-1 & 0\\
\end{array}\right]
=
\left[\begin{array}{*{3}{p{0.5cm}}}
  1 & 0 & 0\\
  0 & 0 & 1\\
  0 &-1 & 0\\
\end{array}\right]
\] 

\[
 W_{rot} = 
 \frac{cos^{-1}(0)}{2} *
 \left[\begin{array}{c}
  -2\\
  0\\
  0\\
 \end{array}\right] 
 =
 \left[\begin{array}{c}
  -\pi/2\\
  0\\
  0\\
 \end{array}\right] 
\]

\section{Linear Trajectory}

To find a linear trajectory, a series of points must be interpolated between the two points. 
The rotations can be found using equation \ref{eq:linear_interpolation}.

\begin{equation}
  R_{i-1, i}(t) = R_{eaa} \left( \frac{t-t_i}{t_{i-1} - t_i} W_{rot} (R^{i-1} [R^i]^T)\right) R^i 
  \label{eq:linear_interpolation}
\end{equation}



\section{Parabolic Blend}

In order to smooth the movement when switching direction, a parabolic movement is used.
The parabola is defined in equation \ref{eq:parabola}.
To find the blend, equation \ref{eq:blend} is used.

\begin{equation}
 P(t-T,X,v_1,v_2, \tau) = \frac{v_2 - v_1}{4 \tau} (t-T + \tau)^2 + v_1 (t-T) + X
\label{eq:parabola}
\end{equation}

\begin{equation}
 p_{i,\tau_i}(t) = P\left( t-t_i, p^i, \frac{p^{i-1} - p^i}{t_{i-1} - t_i}, \frac{p^{i+1} - p^i}{t_{i+1} - t_i}, \tau_i \right)  
\label{eq:blend}
 \end{equation}

The rotation part is found using equation \ref{eq:blend_rotation}.

\begin{equation}
R_{i,\tau_i}(t) = 
\left[ R_{eaa} 
  \left( 
    P\left( 
      t-t_i, 0, \frac{W_{rot}(R^{i-1} [R^i]^T)}{t_{i-1} - t_i}, \frac{W_{rot}(R^{i+1} [R^i]^T)}{t_{i+1} - t_i}, \tau_i
     \right) 
  \right) 
\right] R^i
 \label{eq:blend_rotation}
\end{equation}


\end{document}