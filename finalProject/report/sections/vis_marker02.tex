\subsection{The Second Marker}
The second marker, seen in figure \ref{marker:circle}, contains 3 blue circles and 1 red circle on a green background.
First the image is split into a red, green and blue channel using HSV.
When using HSV to select for colors, Hue is the dominating factor.
The Hue is selected by giving an interval in degrees from the circle.
The Hue is selected as shown in figure \ref{fig:hsv_color_segmentation}.
The saturation is chosen so it selects high values as saturated red, green and blue is desired.
The intensity is not important as the ideal marker gives full intensity of the colors and the real images will give a lower intensity.

\begin{figure*}
\begin{tikzpicture}
 \node {\includegraphics[scale=1]{graphics/hsv} };
 \node[name=hsv, draw, circle, minimum width =5.3cm] at (-4,-1.85) {};
% % % % RED % % % %
\newcommand{\MinHueR}{0} %angle
\newcommand{\MaxHueR}{30}
\newcommand{\MinSatR}{1.6cm} % 0 - 1 => 0 - 2.65
 \draw (hsv.\MinHueR) -- (hsv.center) -- (hsv.\MaxHueR);
 \draw[very thick]   ([shift=(\MinHueR:\MinSatR)] hsv.center) arc(\MinHueR:\MaxHueR:\MinSatR);
% % % % GREEN % % % %
\newcommand{\MinHueG}{70} %angle 
\newcommand{\MaxHueG}{145}
\newcommand{\MinSatG}{1.6cm} % 0 - 1 => 0 - 2.65
 \draw (hsv.\MinHueG) -- (hsv.center) -- (hsv.\MaxHueG);
 \draw[very thick]   ([shift=(\MinHueG:\MinSatG)] hsv.center) arc(\MinHueG:\MaxHueG:\MinSatG);
% % % % BLUE % % % %
\newcommand{\MinHueB}{210} %angle
\newcommand{\MaxHueB}{270}
\newcommand{\MinSatB}{1.6cm} % 0 - 1 => 0 - 2.65
 \draw (hsv.\MinHueB) -- (hsv.center) -- (hsv.\MaxHueB);
 \draw[very thick]   ([shift=(\MinHueB:\MinSatB)] hsv.center) arc(\MinHueB:\MaxHueB:\MinSatB);

\end{tikzpicture}
\caption{HSV used for color segmentation.}
\label{fig:hsv_color_segmentation}
\end{figure*}

The circles are then found by using a Hough transform for circles on the green channel.
The green channel will create a void, and thus edges, where the red and blue circles are found so this is less expensive than finding circles on two colors.
During evaluation of this method, it was found that this method sometimes missed a red or a blue circle.
This could be saved by doing an extra Hough transform on the corresponding channel and find the circle there.

In order to deal with projection, the Hough transform must allow more edges to be viewed as circles.
This means that more points will be found.
In order to remove the wrong points, the surrounding points are analyzed, and if found not to be green, the circle is removed.

The detector was tested to find 4 correct points in the marker in 30/30 images in the easy set and in 50/52 images in the hard set.

In figure \ref{fig:circle_detection} is a normal case from the easy and a worst case example from the hard set shown.
As the robot follows the point, it is not expected to be this bad at any situation so the detector is deemed successful.

\begin{figure}
 \centering
 \begin{subfigure}{0.49\linewidth}
 \includegraphics[width=\linewidth]{graphics/best_case_hough_circle}
 \caption{Normal case.}
 \end{subfigure}
 \begin{subfigure}{0.49\linewidth}
 \includegraphics[width=\linewidth]{graphics/worst_case_hough_circle}
 \caption{Worst case.}
 \end{subfigure}
 \caption{Detecting circles with and without projection.}
 \label{fig:circle_detection}
\end{figure}


