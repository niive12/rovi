\subsection{Conclusion}
An algorithm was found to locate the three markers in an image.
A set of points was then chosen to represent the position of the marker to be used by the tracking algorithms.

The line marker was found using HSV color segmentation for white after which contours were found.
The blobs were then given a grade depending on their compactness and distance to center.
Depending on their grade, the marker was then detected or not.
This was very optimistic and found a fast solution, but also found solutions in images where the marker is not present.

For the circle marker a HSV color segmentation was applied to find a green, blue and red image.
Hough circles was then applied to find the circles surrounded by green.
If the marker was not found in the green image, the circles were also searched for in the blue and red image.

The corny marker was found using SIFT.
The processing time was furthermore improved using cropping on the image depending on the locality of the marker.


%The markers were found to detect the presence of the individual markers.

%Marker one is detected as the center of the white parts of the image.
%Marker two is detected as the center of the circles of the image.
%Marker three is detected as the corners of the marker.