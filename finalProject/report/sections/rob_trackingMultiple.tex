\subsection{Tracking Multiple Points}

When tracking multiple points the same procedure as for one point is used.
Here only the image jacobians and $[du, dv]^T$ are stacked to $2 \cdot k \times 6$ and $2 \cdot k \times 1$ matrix, where $k$ is the number of points tracked.

The implemented algorithm takes in a vector of points of the marker that is wished tracked and a vector with the points in the image frame where these have to be tracked to.
A Dijkstra algorithm is then implemented to find the one-to-one mapping of marker points and their mapping in order to minimize the image displacement vector when tracking the marker.
This algorithm hence performs best for small image displacements as the points tracked then are guaranteed to be mapped to the same point in the image coordinates.
It is hence possible to track all number of points in the image, given that there is at least the same number of points to map the marker to.
Since Dijkstra is used, the mapping with the least image displacements is always found.

The algorithm is however not suitable for large number of points.
This is due to the exponential increase in number of ways to match $n$ found marker points to a corresponding or larger amount of points to map it to.
In this project it was however decided not to use more than four points to track at a time and the algorithm is hence suitable for the given case.

%allows for a vector of points to be used when tracking multiple points.
%This is used to specify the set of locations in the image the different points are to be tracked to relative to the center of the image.

The visual servoing was tested using three points in the markers frame, namely (0,0,0), (0.1, 0, 0) and (0, 0.1, 0).
These were mapped to the image coordinates (0,0), (160,0) and (0, 160) in the image frame.
Figure \ref{fig:robotconfig_ideal_3p}, \ref{fig:toolpose_ideal_3p} and \ref{fig:trackingerror_ideal_3p} show the, as for the one point case, the robot configuration, tool pose and tracking error in pixels when following the marker.
The frame rate, $\Delta t$, for the test of the three marker movements was set to $0.05 sec$.


\begin{figure}[H]
\centering
\includegraphics[width= 0.9 \linewidth]{graphics/robotconfig_ideal}
\caption{Robot configuration when calculating the ideal $(u,v)$ for the marker placement in the image when tracking three points.}
\label{fig:robotconfig_ideal_3p}
\end{figure}

\begin{figure}[H]
\centering
\includegraphics[width= 0.9 \linewidth]{graphics/toolpose_ideal}
\caption{Tool pose when calculating the ideal $(u,v)$ for the marker placement in the image when tracking three points.}
\label{fig:toolpose_ideal_3p}
\end{figure}


\begin{figure}[H]
\centering
\includegraphics[width= 0.9 \linewidth]{graphics/trackingerror_ideal}
\caption{Tracking error in image coordinates when calculating the ideal $(u,v)$ for the marker placement in the image when tracking three points.}
\label{fig:trackingerror_ideal_3p}
\end{figure}

