\subsection{The First Marker}

The first marker, seen in figure \ref{marker:circle}, contains black crosses on a white background.

The marker can be found using the lines in the image.
To do so, all edges must be found.
These are found using a sobel filter and then the number of edges are reduced with the canny edge detector.
\nikolaj{Should I describe in detail how canny edges are found?, I can, but is it relevant?}

The edges are then 

\begin{figure}[H]
\begin{tikzpicture}
 \newcommand{\angleA}{10}
 \newcommand{\radiuA}{1}
 \newcommand{\radiuB}{1}
 
 \FPeval{\angleB}{\angleA + 90}
 
 \node[name=image] at (0,0) {};
 
 \node[scale=0.3,fill=black, minimum width=15cm, rotate={90 + \angleA}, name=lineA] at ([shift=(\angleA:\radiuA cm)] image.center) {};
 \node[name=midA] at ($(lineA.180)!(image.center)!(lineA.0)$) {};
 \node[above] at (lineA.0) {line a};

 \node[scale=0.3,fill=black, minimum width=15cm, rotate={90 + \angleB}, name=lineB] at ([shift=(\angleB:\radiuB cm)] image.center) {};
 \node[name=midB] at ($(lineB.180)!(image.center)!(lineB.0)$) {};
 \node[above] at (lineB.0) {line b};
 
 \FPeval{\rad}{\FPpi/180} %couldn't find the function
 \FPeval{\posX}{round(cos(\angleA * \rad)*\radiuA + cos(\angleB * \rad)*\radiuB,3)}
 \FPeval{\posY}{round(sin(\angleA * \rad)*\radiuA + sin(\angleB * \rad)*\radiuB,3)}
 
 \node[draw, circle]  at (\posX,\posY) {};
\end{tikzpicture}
\end{figure}