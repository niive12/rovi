\documentclass[12pt,a4paper]{article}
\usepackage[utf8]{inputenc}
\usepackage[english]{babel}
\usepackage{amsmath}
\usepackage{amsfonts}
\usepackage{amssymb}
\usepackage{graphicx}

\usepackage{float}
\usepackage{fullpage}

\usepackage{todonotes}
\usepackage{epstopdf}
\usepackage{graphicx}
\usepackage[T1]{fontenc}


\let\ig\includegraphics
\renewcommand{\includegraphics}[2][]{ \IfFileExists{#2}{ \ig[#1]{#2} }{ 
		\IfFileExists{#2.eps}{ \ig[#1]{#2} }{ 
		\IfFileExists{#2.png}{ \ig[#1]{#2} }{ 
		\IfFileExists{#2.jpg}{ \ig[#1]{#2} }{ 
		\IfFileExists{#2.jpeg}{ \ig[#1]{#2} }{ 
		\IfFileExists{#2.gif}{ \ig[#1]{#2} }{ 
		\IfFileExists{#2.ppm}{ \ig[#1]{#2} }{ 
		\missingfigure{  \protect\detokenize{#2} was not found.} 
}}}}}}}
}

\newcommand{\missingequation}[1]{\todo[inline, color = yellow]{Missing equation: #1}}


%%
%% End of file `mypackage.sty'.

\begin{document}


\title{Image Restoration\\ \large{ROVI - Miniproject}}
\author{Nicolaj Iversen and Lukas Schwartz}
\date{9$^{th}$ November 2015}


\maketitle

\pagebreak




\section{Image Restoration}
This report sets to describe the how the \textbf{4} given images from the ROVI lecture are restored to better display the image content.
Each image is separately dealt with in each their section.


\subsection{Pepper Noise}
Pepper noise is when a set of pixels on an image are complete black, without the actual photographed scene has these elements.
This is illustrated in figure \ref{fig:hist_pepper}.

\begin{figure}[H]
\includegraphics[width = 8cm]{graphics/hist.png}
\caption{Histogram of the original ``Image1.png'' showing pepper noise.}
\label{fig:hist_pepper}
\end{figure}



The first image has a lot of pepper noise.
In figure \ref{hist_img1} can it be seen that a majority of pixels are completely black.
This type of damage can be undone, using a median filter.

\subsection{Salt Noise}


\subsection{Gaussian Noise?}


\subsection{Frequency Noise}



\section{Conclusion}


\end{document}